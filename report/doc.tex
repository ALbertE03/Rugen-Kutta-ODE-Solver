\documentclass{article}
\usepackage{graphicx}
\usepackage{mathptmx}
\usepackage{amsmath}

\begin{document}
\section{Entradas válidas}
\subsection{Términos:}
Solo se podrá escribir en términos de '$\textbf{x}$' e '$\textbf{y}$' en cualquier combinación en las funciones y los operadores. También puede usar los números $\pi$ usando la palabra 'pi' y e. 
\subsection{Fuciones:}
\subsubsection{trigonométricas}
\begin{itemize}
    \item $sen(x)$  \textbf{\textit{Representa el seno.}}
    \item $cos(x)$  \textbf{\textit{Representa el coseno.}}

    \item $tan(x)$  \textbf{\textit{Representa la tangente.}}
    \item $cot(x)$  \textbf{\textit{Representa la cotangente.}}
   
\end{itemize}
\subsubsection{Inversas}
\begin{itemize}
     \item $arctan(x)$  \textbf{\textit{Representa la arcotangente.}}
     \item $arcsin(x)$ \textbf{\textit{Representa el arcseno.}}
     \item  $arccos(x)$ \textbf{\textit{Representa el arccoseno.}}
\end{itemize}
\subsubsection{Logarítmicas}
\begin{itemize}
    \item $ln(x)$  \textbf{\textit{Representa el logaritmo natural o neperiano.}}
    \item $log(x)$  \textbf{\textit{Representa el logaritmo en base 10.}}
\end{itemize}
\subsubsection{Exponenciales}
\begin{itemize}
    \item $f(x)^{g(x)}$ 
\end{itemize}
\subsection{Algebraicas}
\begin{itemize}
    \item $p(x)$ \textbf{\textit{Representando cualquier polinomio}}
\end{itemize}
\subsection{Operadores}
\begin{itemize}
    \item $+$  \textbf{\textit{Representa la suma.}}
    \item $-$  \textbf{\textit{Representa la resta.}}
    \item $*$  \textbf{\textit{Representa la multiplicación.}} \begin{itemize}
        \item $\#(...)$ y $(...)\#$ \textbf{\textbf{Será tomado en cuenta como $\#$*(...) y (...)*$\#$}}\\
        \item  $\#x$ y $x\#$ \textbf{\textit{Será tomado en cuenta como $\#$*x y x*$\#$}}\\
        \item $x(...)$ y $(...)x$ \textbf{\textit{Será tomado en cuenta como x*(...) y (...)*x}}\\
        \item  \textbf{EN CUALQUIER OTRO CASO TENDRÁ QUE INDICAR LA MULTIPLICACIÓN EXPLICITAMENTE.}
    \end{itemize}
    \item $/$  \textbf{\textit{Representa la división.}}
    \item \textasciicircum{} ó $**$ \textbf{\textit{Representan la potenciación.}}
    \item (  \textbf{\textit{Representa el paréntesis de apertura.}}
    \item  )  \textbf{\textit{Representa el paréntesis de cerrado.}}
\end{itemize}
\section{Runge-Kutta}
Se implementó el Runge-kutta de orden 4 con la siguiente tabla de butcher:
... lo pongo después...

despues viene lo de diego...
\end{document}
